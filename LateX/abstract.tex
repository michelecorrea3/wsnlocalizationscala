\IEEEcompsoctitleabstractindextext{%
\begin{abstract}
Localization of nodes in wireless sensor networks (WSNs) is important to context-aware and position-dependent applications; data are generally meaningless without a known location. Many algorithms exist for localizing nodes using RSS; however, a detailed quantitative comparison of these algorithms has not yet been published. In this paper, we present such a quantitative comparison of algorithms which use RSS as a ranging method and present a localization software framework called Senseless. We also conducted a survey about the influence of the orientation of a node, thus the radiation pattern. Our study finds that the received power is not equal in all directions and that no single best algorithm for localization exists to date. Each algorithm has a different purpose and diverse properties. Learning from these algorithms and techniques, the correct algorithm can be chosen for the correct environment and a more advanced localization system is feasible. Using this framework, we implemented a couple of centralized algorithms; Trilateration, Min-Max, Centroid Localization and Weighted Centroid Localization.
\end{abstract}

\begin{IEEEkeywords}
Wireless Sensor Network, WSN, RSSI, localization, TelosB, TinyOS, Senseless.
\end{IEEEkeywords}
}